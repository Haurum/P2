\chapter{E-mail korrespondance med Jens Børsting}
\textbf{Hvordan foregår en træningsgang af o-løb?}\newline
Man starter med at få en skovtilladelse hvor løbet skal foregå.
Træneren beslutter hvad der skal trænes og planlægger løbet i et computer program der hedder ”Condes”. Her lægges baner og tegnes kort. Herefter udskrives kort for alle de baner der skal løbes. Dette inkluderer et kort til udsætning af alle poster. Det meste af dette arbejde kan gøres hjemmefra. Dog er det ofte sekretæren i klubben der printer kortene.
Typisk dagen før træningen hentes alle poster i klubhuset og sættes ud i skoven.
På træningsdagen mødes alle løbere og får instruktion i løbets momenter i dag og baner fordeles alt efter niveau og kondition. Typisk vil der være 3-7 baner at vælge imellem.
Løberne sendes i skoven. Alle løber med en elektronisk brik der registrere hvornår man har været ved her post.
Ved hjemkomst får man en udskrift over hvilke poster man har været ved og hvor lang tid der er gået mellem hver post (stræktider)
Man har så mulighed for at gennemgå løbet ved at snakke med andre løbere og trænerne. Her diskuteres vejvalg og hvad der gik godt og hvad der gik mindre godt. Stræktider sammenlignes. Vejvalg foregår udelukkende efter hukommelse og det er ikke muligt at se forskel i hastighed på kortere stræk end hele strækket mellem 2 poster.
Hvis løberen ikke er sikker på hvor han/hun har været kan det være meget svært at analysere hvad der gik galt.
Dagen efter løbet skal alt udstyret der er placeret i skoven hentes ind igen og pakkes ned i klubhuset hvis det ikke skal hænges til tørre først.

\textbf{Hvordan evaluerer trænere deres løbere? (Hvad vurderer han på, og hvordan observerer han dette?)}\newline
Eneste mulighed træneren har er at snakke med løberen efter løbet og ud fra stræktider og løberens hukommelse af vejvalg diskutere hvad der virkede og hvad der ikke gjorde.

\textbf{Hvordan evaluere respondenten sin egen tur? (hvordan kan respondenten selv se fremskridt eller fejl ved sin egen træning?)}\newline
På samme måde som med træneren kan løberen læse sine stræktider og evt. sammenligne med andre løbere der har løbet sammen stræk. Det kræver dog at disse løbere er til stede. Ved de ugentlige træninger bliver stræktider ikke offentliggjort, så man kan ikke sammenligne online efter løbet. Ved større løb bliver alle stræktider offentliggjort på nettet og man kan sidde hjemme i ro og mag senere og analysere sit løb i forhold til andre. Det er vigtigt at man kan sammenligne med andre da der er stor forskel på løbsterræn og dermed hastighed i terræn ved forskellige løb. Man kan dog sammenligne med dem man normalt lige op med løber med og kan se om man har løbet hurtigere eller langsommere end den på dette løb i forhold til tidligere løb.

\textbf{Hvordan kan denne evaluering gøres mere præcis, eller endda indkludere flere aspekter af løbet?} \newline
Ved at have gps tracking som kan følges live under løbet kan træneren se hvordan løbet forløber og nemmere snakke vejvalg og andet teknisk træning efter løbet. Evt. kan vejledning gives under løbet, hvis løberen er faret helt vild. Tracks kan sammenlignes med andre for at se hvor på strækket der har været fart på og hvor der ikke har. Samlet afspilning af løberne fra en post til en anden kan give et godt billede af udviklingen af løbet.
Det at have gps med på løbet kan også give mulighed for at give elektronisk tilbagemelding om at man er ved posten, kan man reducere arbejdsmængden ved udsætning og indhentning af poster. Samtidig er løberen uafhængig af hvornår der er poster i skoven og kan løbe løbet når det passer ham (dog skal der være løbstilladelse i skoven).

\textbf{Hvilke redskaber og metoder kender respondenten til, og hvilke bliver generelt set brugt i træningen?}\newline
Kort, stræktider og hukommelse er det normale. Sammenligning af stræktider med andre ved løbets afslutning eller ved større løb online hjemme i de efterfølgende dage.
Nogle få kender til Travel Tales og PDFmaps, men det bliver stort set aldrig brugt da det ikke giver en samlet løsning.
TracTrac og lignende systemer bliver kun brugt at landsholdsløbere og til nogle få eliteløbere til de store stævner.

\textbf{Hvordan mener respondenten, at en træningsgang kan blive optimeret? (Hvilke evaluerings punkter er mangelfulde)}\newline
Ved at kunne sammenligne gps tracks kan man nemmere vurdere om hvilke vejvalg der har virket og hvor det er gået galt/langsomt. Her kan man se om det kunne betale sig at løbe udenom eller det var nemmere at løbe igennem en tæthed eller over en stejl bakke mv.