\chapter{Interview transskribering}
Dette er et interview med Claus Bobach, foretaget af Frederik, Mark og Søren. 

\textbf{Søren:} Vi går på software på Aalborg Universitet, og har valgt et projekt som skal omhandle små foreninger, og Jens Børsting foreslog at tage kontakt til dig.\newline
\textbf{Frederik:} Lige nu er vi i en fase, hvor vi forestiller os en form for hjælp med jeres træning, med noget udstyr og software. Så vi vil gerne have dig til at fortælle lidt om hvordan en træningsgang forløber. Så kunne vi senere snakke om du har nogle forestillinger om, hvad der kunne hjælpe til træning, af software. Så hvis du kan forklare os hvordan en træningsgang forløber, helt fra i planlægger den, til udførelse og efter udførelsen.\newline
\textbf{Claus:} Vi starter med et blankt kort, på et program der hedder Ocad, der har vi så et program til at lægge lagene ovenpå, Condes, et forholdsvis simpelt program som er udvidet så mange gange at det er blevet rigtig godt. Her har vi mulighed for at lægge en simpel bane, eller noget mere kompliceret som at fjerne dele af kortet. Disse to programmer har så inden for de seneste år kunnet arbejde sammen, og de lag der så ligger i Ocad er delt om i farver og symboler, og i Condes kan man så bestemme hvad af det man vil have med. Så kan man eventuelt vælge kun at se kurvebilledet, når de to programmer arbejder sammen. Hvis ikke man har en cad-fil, så kan man bruge JPG filer, men så har man ikke samme mulighed for at ændre kortet. 
Så når jeg har lavet banen på et kort, så sender jeg det videre til en som printer det, og sender det til dem som sætter posterne ud. Jeg har så en stak kort til de folk som møder op. Så løber folk ellers ind i skoven og kommer hjem igen. \newline
Til de fleste træninger står folk selv for at skulle tage tid, hvis det er noget de vil. På de ugentlige træninger har vi ikke noget tidstagnings udstyr. Det har vi dog til de lidt mere specielle træninger, blandt andet traditionsløb og sådan, hvor vi har elektroniske enheder ude ved posterne. Her har vi så en enhed vi kan se det hele på, og printe ud fra. \newline
\textbf{Frederik:} Kan du sige lidt om, hvor lang tid det tager at lave kort og sætte poster ud?\newline
\textbf{Claus:} Jeg bruger normalt en time eller halvanden på at lave de tre baner vi løber på til almindelig træning, men vi har en gruppe på otte som skiftes til at sætte posterne ud. At sætte posterne ud tager typisk et sted mellem halvanden til to timer. Så der er selvfølgeligt noget forberedelse. Også i form af at hente posterne igen, men det prøver vi på at gøre i forbindelse med træningen. For eksempel i dag, der løb jeg bare som den bagerste og samlede posterne ind.\newline
\textbf{Frederik:} Du sagde, at i kunne vælge kun at have højdekurverne på kortet. Hvordan er dette relevant?\newline
\textbf{Claus:} Blandt andet her om onsdagen, der prøver vi på at øve forskellige teknikker, nu hvor orienteringsløb er rimelig komplekst, så vi deler det lidt op, så man kan øve forskellige ting. Det er der vi bruger det. Kurvebilledet er en af de ting der er vigtige at træne og læse. Så for at få det mere simpelt til træningen, har vi nogle gange kun kurvebilledet på kortet.\newline
\textbf{Søren:} Når i skal evaluere hvordan i løber, hvordan gør i det?\newline
\textbf{Claus:} Det optimale er selvfølgeligt at vi havde enheder på hver enkelte post hver gang, men det tager allerede halvanden time til to timer at sætte poster ud, og hvis vi så skal have enheder med hver gang, tager det en hel time mere. Og det at løbe bagerst og samle posterne ind, vil heller ikke være lige så nemt. Men der er to muligheder, som man også kan kombinere, men at bruge tiderne mellem de enkelte poster eller at bruge GPS-ur er de to muligheder der typisk bruges. Så de fleste scanner bare kortet ind og lægger GPS-dataen oven i.\newline
\textbf{Frederik:}  Ved du hvilket program de bruger til det? Altså til at samle GPS dataen.\newline
\textbf{Claus:} Der er en orienteringsløber oppe i Stokholm, eller sådan noget, det hedder Quickroute. Hvor han henter alt data hjem der ligger i gps uret, det vil sige puls og højde osv. Så man kan få graferne sammen og få det visuelt på kortet at hvis man eksempelvis vælger hastighed, så har løberen en skala fra rød til grøn.\newline
\textbf{Frederik:} Så man kan se hvor man har løber hurtigt og langsomt?\newline
\textbf{Claus:} Ja. \newline
\textbf{Frederik:} Okay. Kan den replay det? Altså så man kan se hvor langt man kom?\newline
\textbf{Claus:} Ja, jeg har faktisk ikke prøvet funktionen, men ja. Jeg tror han har lavet det og lavet det i forbindelse med Google Earth. Hvor Quickroute ligger o-kortet ind i Google Earth også laver den en replay der. \newline
\textbf{Frederik:} Okay, det tror jeg vi skal have kigget nærmere på hvad det er for noget. \newline
\textbf{Claus:} Der er helt klart også noget interessant der.\newline
\textbf{Frederik:} Når i nu ikke har tidtagning med der ude, hvordan evaluere i så bagefter? Det er måske bare lidt for at få erfaringen og holde formen ved lige?\newline
\textbf{Claus:} En ting er jo at nogle har deres GPS ur og selv går hjem og evaluere, men vi opfordre selvfølgelig også folk til at snakke sammen. Bare det at have set hinanden i skoven, kan man godt fornemme hvilke udfordringer de andre har haft. Det man har gjort helt tilbage fra starten af, er jo man har sat sig ned og snakket om hvad gjorde du og hvad gjorde jeg. \newline
\textbf{Søren:} Så vidt som de kan huske selvfølgelig.\newline
\textbf{Claus:} Ja selvfølgelig. \newline
\textbf{Søren:} Hvad med andre redskaber der kan bruge til at evaluere folk i deres træning?\newline
\textbf{Frederik:} Er der f.eks. nogle der bruger deres mobil telefon?\newline
\textbf{Claus:} Jeg tror det er mest til GPS urene, der er nok nogle få der bruger Endomondo, men ellers tror jeg ikke det er noget der er så udbredt. Det er så det Quickroute, hvor det er meget for den individuelle løber. Der er også nogle andre programmer, hvor arrangørerne, altså efter en konkurrence, at de lægger et kort op, hvor alle så kan lægge deres GPS rute ind oven i. De bliver desværre ikke brugt så voldsomt meget lige i forløbet, det var meget populært for en 3-4-5 år siden. Det var som om der var et enkelt program som var rigtig brugervenligt, som folk brugte, men det var der så åbenbart ikke økonomi i, så det svenske forbund valgte at begynde at bruge noget andet, det har man så ikke rigtig for arrangører og løber ind til at støtte op om. Det kræver så, for virkelig at få noget ud af det, skal man have 20-30\% af løberne til at bruge det. \newline
\textbf{Frederik:} Ja okay, det er vigtigt de alle sammen er med på det. \newline
\textbf{Claus:} Ja, så derfor kan det godt være vigtigt som arrangør, hvor vi nu skal arrangere et løb her om 2½ uge for 1.500 løbere, hvis vi f.eks. som vi nogle gange har gjort lægger kortet op i et af de der programmer, så er det vigtigt at vi reklamere for det, ellers bliver det jo bare udvasket. \newline
\textbf{Frederik:} Kan du huske hvad det hed?\newline
\textbf{Claus:} Altså det de stoppede med at bruge, det hed RunOWay.\newline
\textbf{Søren:} Vi har også hørt om TracTrac, om det er noget der er udbredt? Men som vi kunne forstå, var det forholdsvist dyrt. \newline
\textbf{Claus:} Det bliver mest brugt til at live rapportere. Enten hvis vi sidder der hjemme, når der er VM et eller andet sted ude i verden, så kan vi se med hvor. De bruger det også til at have en stormskærm på pladsen, når der er konkurrencer. \newline
\textbf{Frederik:} Det er mest til større events?\newline
\textbf{Claus:} Ja det er det. Jeg er lidt i tvivl om landsholdet måske, har brugt det lidt.\newline
\textbf{Frederik:} Jeg tror de bruger det, det mente Jens Børsting i hvert fald.\newline
\textbf{Claus:} Det tror jeg faktisk også. Det udstyr der bliver brugt til de store konkurrencer herhjemme, har landsholdet jo til dagligt. Så jeg tror måske de bruger det lidt, men jeg ved faktisk ikke hvor meget.\newline 
\textbf{Frederik:} Jeg tænkte om du måske havde nogle idéer til områder, hvor du kunne forestille dig der var et eller andet en eller anden form for optimering i forhold til noget software du havde tænkt over? Inden vi snakker vores idé. \newline
\textbf{Claus:} Vi har et enkelt problem i øjeblikket, men det tror jeg også er lidt et spørgsmål om vilje herfra. Når vi arrangere et o-løb herfra, så for at få data ind fra skoven, ikke GPS data, men bare tidsdata, bare fra en enkelt post eller to, så er vi meget afhængige af… Det er vi nød til at kable ud til. Altså, at vi decideret har et fysisk kabel, og det begrænser lidt muligheden for hvor langt udefra man kan få meldingen. Så der vil vi jo gerne have en enhed som kunne sende det, eller nogle programmer der kan håndtere det, for jeg tror nok at enheden findes. Jeg tror nok de bruger det nede i Sønderjylland, men de bruger et andet program der skal modtage det. Så det, ja, det er en af vores udfordringer i øjeblikket. \newline
Mark: Men er det i forbindelse med at få data direkte, eller når i kommer tilbage og skal evaluere.\newline
\textbf{Claus:} Det er med det samme. Vi har en mand på pladsen der sidder og speaker om hvordan det går. Lige nu bruger vi det bare sådan at han har en forvarsling på hvem der kommer næst, altså så han kan forberede sig, men specielt på de lange distancer ville det være rart at have en post halvvejs han kunne speake nogle tider fra.\newline
\textbf{Frederik:} Ja men vi kunne måske snakke lidt om hvad vi har tænkt på. 
En af de første ting vi tænker, var virtuelle poster, hvor du får en form for lyd, når du er i nærheden. Men vi fandt noget der faktisk var det vi havde forestillet os, og hvis det er ude, og det ikke bliver brugt, så tænkte vi at det nok ikke var noget der var så aktuelt.I stedet, så vil vi gerne prøve at lave noget der ligne TracTrac, hvor du bruger din mobil som GPS-sender, for så vidst vi har forstået, så er der ikke nogle af GPS-urene der kan sende data direkte. Nu må vi se hvor præcise GPS’erne i mobilerne er, men så ville det jo være noget man kunne gøre live, og så ville det være noget hvor man, igen som du siger, lave en eller anden form for program som arrangørerne laver, og så få lagt kort ind, og få samlet alle folk ruter og sådan nogle ting, så de kan sammenligne, meget ligesom TracTrac, men så prøve at gøre det lidt billigere fordi det er via mobilen. Men det ved jeg ikke hvad du tænker om?\newline
\textbf{Claus:} Jamen det tror jeg da helt klart kunne være interessant.\newline
\textbf{Frederik:} Er det noget til sådan en almindelig træning som i dag, tror du der er nogle der ville bruge det der, eller ville det være til arrangerementer?\newline
\textbf{Claus:} Altså det live mæssige ville vi nok ikke bruge til træning, men jo hvis det er noget man nemt bagefter ville kunne tage frem, så tror jeg da helt sikkert at det… \newline
\textbf{Frederik:} Altså jeg tænkte at hvis man nu havde en projekter op her på væggen, så kunne man…\newline
\textbf{Claus:}  Så kunne man bruge det umildbart efter ja.\newline
\textbf{Frederik:} Det var i hvert fald det vi havde i tankerne\newline
\textbf{Claus:} Jamen det er da rigtigt, det tror jeg godt man kunne få nogle til at bruge\newline
\textbf{Frederik:} Så er i også fri for de der tidtagninger ude i skoven, så behøver i ikke dem. Så kan i jo bare se.\newline
\textbf{Claus:} Nej nej\newline
\textbf{Frederik:} Det er vores tanke lige nu\newline
\textbf{Claus:} Der er nogle af de der, der har gjort noget lignende. Det vi bruger posterne ude i skoven til, er jo at få mellemtider. Der er nogle af de der programmer der faktisk har kunne gøre det ved hjælp af GPS’en. \newline
\textbf{Frederik:} Så den registrere om man er kommet tæt nok på posten, og at man så regner med at man har fundet den.\newline
\textbf{Claus:} Ja, et eller andet estimat af hvornår folk har været ved posten, så man ved hvor lang tid man har brugt mellem posterne, uden at man skal sidde og.. ja..\newline
\textbf{Søren:} Jeg tror egentlig også vores primære ide, var at man kunne sammenligne de vejvalg og de stræk man tager, så vi kunne tage fra en post, og…\newline
\textbf{Claus:} Ja ja\newline
\textbf{Frederik:}  Jeg havde ikke tænkt over den med at det tager længere tid for jer at sætte de poster ud der skal brik til, end dem der ikke skal, men det er klart. \newline
\textbf{Claus:} Men det gør det. Fordi det vi sætter ud nu, det er skærme i den her størrelse, det koster jo ingenting at have med. \newline
\textbf{Frederik:}  Så du skal selv ligesom, hvad skal man sige. Der er ikke noget papir for at validere at du har været derude. \newline
\textbf{Claus:} Nej, vi hænger heller ikke stifteklemmer derude, for det er jo også. Hvis du løber rundt med nogle poster med en snor på, og der hænger noget tungt ude for enden, så i løbet af 5 minutter, så er de snore viklet sammen.\newline
\textbf{Frederik:} Jeg tror heller ikke vi har så meget mere.\newline
\textbf{Søren:} Nej, egentlig ikke.\newline
\textbf{Frederik:} Så er det pizza tid!\newline 
