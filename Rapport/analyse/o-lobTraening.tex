\chapter{Typisk o-løbs træning}
I dette afsnit vil en typisk o-løbs træning blive beskrevet, og der bliver set på nogle af de problemer o-løbere støder på i forbindelse med en træning. Afsnittet er udarbejdet ud fra information af Jens Børsting.

Før en træningssession kan starte, skal der først søges tilladelse hos Naturstyrelsen, til den skov, hvor der skal løbes. \newline
Til hver træning beslutter træneren hvilke fokuspunkter, der skal trænes, eksempelvis tekniske fokuspunkter, som højdekurvelæsning, eller mere fysisk som konditionstræning. Herefter planlægges løbet i et computerprogram som f.eks. Condes, hvor de enkelte ruter/baner tegnes på orienteringskort. Desuden udarbejdes der også et orienteringskort, som bruges til udsætning af poster til træningen. Det meste af dette arbejde, kan gøres hjemmefra, dog er det ofte sekretæren i klubben, der står for at printe orienteringskortene. Hele planlægningen af træningen tager omkring to timer. Dagen før træningen, bliver alle poster typisk hentet i klubhuset og sat ud i skoven, denne proces tager omkring to timer. Der er flere forskellige typer af poster. De mest simple er en skærm, altså bare en farvet stofkasse, der blot indikere hvor posten er. Der findes også en større og besværligere udgave, der er en pind i jorden på ca. 1m, med en skærm omkring og en elektronisk aflæser på toppen, som løberne kan bruge Emit-brikker til. Det tager cirka en times ekstra arbejde hvis man vælge de større elektroniske poster.\newline
På dagen mødes alle løbere og får instruktion i løbets fokuspunkter. Herefter uddeles baner alt efter niveau og kondition. Der er typisk 3-7 baner at vælge imellem. Løberne bliver derefter sendt ud i skoven, med en kort, kompas og evt. Emit-brik hvis det er en mulighed. Når løberne er færdige med deres tur, får de en udskrift over hvilke poster de har været ved, og hvor lang tid der er gået mellem hver post, også kaldet stræktider, hvis Emitbrikkerne har været taget i brug.
Efter løbeturen, har løberne mulighed for at evaluere deres tur, ved at snakke sammen med andre løbere, og sammenligne vejvalg og stræktider. Vejvalg foregår udelukkende efter hukommelse og det er ikke muligt at se forskel i hastighed på delstræk, kun hele stræk mellem 2 poster. 
Efter træning, eller dagen efter, skal alle poster samles ind igen og pakkes ind i klubhuset.\newline


\textbf{Problemer}\newline
Der bliver brugt mange timer på forberedelse af en træning eller et løb, og derudover skal det hele pakkes sammen igen, da alle posterne skal sættes ud på ruten, og samles ind igen. \newline
Problemet er at der skal folk til at gøre det, og i små frivillige foreninger, er der ikke nogen der kan blive betalt løn for at gøre det, men man skal udelukkende satse på frivillige der gider at tage ansvaret for det. \newline
Derudover er det et endnu større arbejde, hvis der ønskes en form for tidtagning på posterne, da de elektroniske poster tager længere tid at stille op og samle sammen. 

Problemet med evaluering af træningen for både træner og deltagere, er problemerne med at kunne se tider på delstræk, og vejvalg. Bare fordi to personer har løbet cirka lige hurtigt mellem to poster, behøver det ikke at betyde at de begge har fundet den samme gode vej. Det kan fx være at den ene var hurtigere på den første del på grund af vejvalgt, mens den anden var hurtig på den sidste del, og det i virkeligheden ville være meget hurtigere at vælge en kombination af de to ruter.

Hvordan en typisk o-løbstræning foregår, afhænger af klubben og løberne, men som en generel hovedregel, kan det siges at der inden en træning er blevet lavet 3 eller flere ruter der kan løbes. Hvor meget snak om løbet og erfaringsudveksling der foregår, er hos amatørløberne helt op til løberne selv, og hvis lette værktøjer ikke er der til det, så kan det hurtigt blive meget begrænset hvor let det er at snakke om sine ruter.
