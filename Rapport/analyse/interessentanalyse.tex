\section{Interessentanalyse}
I dette afsnit vil indflydelses- og medvirken matrixen blive anvendt, for at undersøge og prioritere interessenter for dette projekt. Dette undersøges for, at finde ressourcepersoner som gruppen kan gøre brug af gennem interviews. Disse personer vil være respondentgruppen til undersøgelse af den initierende problemstilling.
Jens Børsting har hjulpet til med at finde interessenter til projektet, og har senere hjulpet med et interview.

\subsection{Aktører}
I dette projekt har gruppen fundet fire aktører, som gruppen mener er relevante til projektet. De forskellige aktører er henholdsvis o-løberne, trænerne, sportens forbund og foreninger og frivillige. 

\subsubsection{O-løbere}
For at den enkelte o-løber skal kunne forbedre sig, er det vigtig at kunne sammenligne løberens rute detaljeret med andre. Lige nu er tiderne mellem hver post (stræktiderne) det eneste der kan sammenlignes og analyseres på. Her er det interessant for løberen at kigge på vejvalg og hastighed mellem posterne, og endda helt ned til de forskellige faser af delstrækkene. Til dette mangler der mere detaljeret data om løbet. Problemet håndteres i dag ved at sammenligne skemaer med stræktider og hvis muligt manuelt indtegne vejvalg på kortet efter løberens hukommelse. Derfor har den enkelte o-løber interesse i dette projekt, da der arbejdes med afvikling og opfølgning af træningen. 

\subsubsection{Træneren}
Træneren har interesse i at mindske arbejdstiden brugt på forberedelse og planlægning af løb, da dette giver mere tid til træning af de enkelte løbere. Samtidigt vil træneren gerne kunne analysere den enkelte løbers tur detaljeret, ved at sammenligne løberens rute med andre løberes rute. Hvis løberen ikke kan huske hvor vedkommende har løbet, eller var faret vild, har træneren svært ved at give sikker og brugbar kritik, da det ikke kan ses på tiderne, præcis hvor den enkelte løber har været. Trænere har derfor interesse i et værktøj som kan hjælpe med planlægning og afviklingen af træningen, samt evaluering af den enkelte løbers tur.

\subsubsection{Forbund og foreninger}
O-løbernes forbund hedder Dansk Orienterings Forbund, også kaldt for DOF, som ligger under Dansk Idrætsforbund, DIF. Der er i alt 76 foreninger i DOF, med lidt under 7.000 medlemmer\citep{DIF}. DOF er med til at drive landsholdet, samt står for talentudviklingen inden for orienteringsløb. Dette gør DOF og foreningerne til interessenter i dette projekt, da de bl.a. ønsker deres løbere skal blive så gode som mulige. \citep{DIF}

\subsubsection{Frivillige}
I foreninger er frivillige vigtige, da 80\% af alle lokale foreninger er udelukkende benytter frivillig arbejdskraft. Derudover er mange af de frivillige engageret i o-løb, da de er villige til at lave et stykke ubetalt arbejde, for at holde gang i klubben. \citep{frivillig}




\subsection{Prioriteringen}
Indflydelse- og medvirken matrixen deles op i fire rum, hvor hvert rum er skaleret efter ”indflydelse på projektet” og ”afgørende medvirken i projektet”. Dette giver rum for diskussion af de enkelte interessenter, for at undersøge hvilke interessenter, der skal tages kontakt til, for at få svar på den initierende problemstilling.
Gruppen har i denne rapport meget få interessenter, hvilket medfører, at de enkelte interessenter hurtigt bliver ressourcepersoner.

\textbf{Eksterne interessenter} har hverken en stor indflydelse eller en stor medvirken i projektet. Det betyder at der ikke nødvendigvis skal tages stor hensyn til denne gruppe af interessenter.\newline 
\textbf{Den grå eminence} har stor mulighed for at påvirke beslutninger i projektet, men deres medvirken er ikke særlig stor i projektet. Dette vil ofte være en person med en magtfuld stilling eller ledelsesposition. \newline
\textbf{Gidsler} har ikke stor mulighed for at træffe beslutninger i projektet, men deres aktive medvirken er vigtig for projektet.\newline
\textbf{Resourcepersoner} har både stor indflydelse og stor medvirken, da det er denne gruppe der skal inddrages i projektet. Denne gruppe har erfaring eller faglige kompetencer indenfor området. \newline

O-løbere er i dette projektet sat som ressourceperson, da de kan give råd og vejledning til, hvordan deres træning og løb fungere, samt undersøge om der er ting der kan forbedre deres løb. Dette gælder både inden- og efter løbet.

Trænere er sat som ressourceperson for projektet, da de ligesom o-løberne har et stort indblik i hvordan orienteringsløb fungere, og hvordan det kan optimeres eller forbedres. 

DOF og foreningerne er i dette projekt grå eminence, da de kan have en indflydelse på projektet. Herudover kan de have nogle krav og regler til en løsning. Deres medvirken er dog ikke nødvendig for at projektet skal kunne blive en succes.

I dette projekt har gruppen valgt at placere de frivillige som gidsler. De frivillige har ikke nogen indflydelse på hvordan projektet bliver lavet, men de laver stadig et stykke arbejde, som er værd at tænke over i løsningen.    

\includegraphics[width=0.70\textwidth]{billeder/Prio}
\vspace{0.20cm}

\subsubsection{Opsummering}
O-løbere og trænere er sat som ressourcepersoner, derfor vil gruppen trække på den viden som disse grupper kan give. Dette vil ske i form af interview og e-mail korrespondancer. Disse vil desuden være målgruppen for projektet.