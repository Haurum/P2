\chapter{Eksisterende løsninger}
I dette afsnit vil de nuværende IT-løsninger blive forklaret, som har indvirken på problematikker vedrørende o-løbstræning. 

\section{Endomondo}
”Den succesrige danske løbeapplikation, Endomondo, er blevet solgt til et stort, amerikansk selskab”, sådan lyder nyhederne om denne applikation til Smart-phones, som bruges af omkring 25 millioner mennesker. \citep{ENDO}

Endomondo er en applikation som bruges til løb, hvori en træningsplan kan udformes. App’en vil her have to versioner, en gratis og en udvidelses-version som kan købes. Endomondo bruger Google Maps som kort i deres app\citep{ENDOMAPS}.Funktionaliteten der her vil blive beskrevet er for gratis versionen: Hovedmenu med otte funktioner:
\begin{itemize}
\item Newsfeed, som bruges til at kommunikere med andre brugere af Endomondo, hvor brugere kan opmuntre, samt udfordre hinanden til
\item Notifikationer, hvor brugeren kan se sine udfordringer og lignende fra andre brugere, her kan brugeren acceptere eller afslå udfordringer.
\item Historik, hvor tidligere løb oplagres til genvisning og statistik.
\item Kort, hvor brugeren ved hjælp af mobilens GPS-enhed kan få et kort over ruten der er løbet. Kortet er standardiseret, og viser personens placering og rute.
\item ”Opgrader-nu”, hvor gratis versionen kan betales til opgradering.
\item Venner er en funktion, hvor brugeren kan overvære venner fra sociale medier.
\item Træningsplanen bruges til at indstille app’ens hjælpefunktioner til træning, hvor brugeren selv sætter mål for træningen, intensiteten ved træningen, og hvilken typen af træning der udøves.
\item Indstillinger er en funktion der findes i alle programmer, hvor de generelle indstillinger for programmet kan tilpasses brugeren.
\end{itemize}
Dette er en velfungerende app til formålet, løb og træningsplan, men i det en O-løber skal bruge visningen af løbet på et kort som brugeren selv har fra det specifikke O-løb, kan dette blive problematisk, da brugeren ikke selv kan sætte kort ind i programmet. 


\section{EMIT-brikker}
En EMIT-brik, er et elektronisk apparat, der kan registrere hvilke poster der besøges. EMIT-brikken sidder fast på fingeren, ved hjælp af et elastik. \newline
Måden en EMIT-brik fungere på, er at den stilles på en startpost ca. 5 sekunder før løbet bliver sat i gang. Dette vil genstarte EMIT-brikken, således at den kan notere de rigtige tider. 
Ved hver post, der skal besøges, er der en kontrolpost, der ligner startposten, som EMIT-brikken skal lægges på i ca. ½ sekund. Dette vil registrere hvor lang tid der er gået mellem forrige post og nuværende post. \newline
Til sidst i løbet, vil der være en målpost, hvor EMIT-brikken endnu engang skal lægges på, således at den sidste tid bliver noteret. 
Herefter skal EMIT-brikken afleveres til de ansvarlige, som derefter vil give løberen en udskrift af tiderne. Disse tider kan så derefter anvendes til sammenligning og diskussion med andre løbere.\citep{OOK}

\section{QuickRoute}
Igennem gruppens interview, beskrev Claus, at QuickRoute er en eksisterende løsning på kortlæggelse og rute visning. Denne løsning kommer på bekostning af et Garmin ur, eller andre GPS enheder, som kan generere en GPX fil over ruten. I QuickRoute kan et kort fra Ocad lægges ind, og ved hjælp af Garmin ure kan en route blive vist, hvor forskellige parametre kan blive afbildet. Hastighed, minutter per kilometer, hjertefrekvens, højdemeter og afvigelse fra retningen mellem to punkter kan blive afbildet. Dette bliver vist ved en farve-kode der følger ruten, og gennem hele ruten vil denne farvekode variere efter hvert interval af GPS-signal. Med den rigtige serie af Garmin ure, kan der tilmed tages tid på, hvornår en post er nået, så en statistisk model kan beskrive tiderne mellem posterne. Hvis musen holdes over et punkt på ruten, kan følgende information om punktet vises: Klokkeslæt, tid brugt i alt på rute, samlet distance løbet til det punkt, nuværende minutter per kilometer, nuværende hastighed i km/t, nuværende hjertefrekvens, nuværende højdemeter, nuværende afvigelse mellem to punkter, nuværende længde- og breddegrader. Mange af disse informationer kan afbildes statistisk både grafisk og på et histogram \citep{QR}.\newline
Til at toppe alt dette af, kan det efterfølgende integreres på Google Earth, så der kan ses en 3D model af ruten der er brugt. 
Denne løsning er meget gennemført, hvor mange funktioner kan benyttes, dog kan antallet af funktioner og parametre virke overflødig for en amatør o-løber, og der kan være for mange informationer. For en rutineret eller professionel o-løber vil denne løsning give et godt indblik i, hvordan løbet er foregået og hvor personen kan udvikle sig.


\section{TracTrac}
TracTrac er en samlet løsning til livetracking med replayfunktion af forskellige sport events. Til dette bruges TracTrac's egne GPS enheder. Disse er ca. på størrelse med en cigaretpakke og vejer 113 gram.\newline
TracTrac bruges til o-løb så der live kan følges med i hvor o-løberne er ude i terrænet, som f. eks. Til stævner og konkurrencer. Derudover bruges TracTrac i høj grad til at analysere de enkelte løberes ture efter de har løbet, da TracTrac har en velfungerende replay funktion. 

Når TracTrac skal bruges i forbindelse med o-løb, laves der først et o-løbskort som oploades til TracTracs servere. Derefter indsættes præcise punkter på kortet som repræsentere hvor hver enkelt post ligger, så TracTrac kender positionerne på alle posterne. Herefter skal hver enkelt GPS enheds nummer sættes sammen med en løber, så det bliver tydeligt hvem der løber hvor. Under løbet sender GPS enhederne deres position til serverne som viser disse på kortet. Alt dette sker live. En af funktionerne som gør TracTrac ekstra brugbar i forbindelse med o-løb, er dens mulighed for at flytte løbere tilbage til start og afspille deres tur samtidig, så man kan se præcist hvordan de løb i forhold til hinanden, selvom de i virkeligheden startede forskudt. Dette kan også gøres selvom løbet er live. En løber der er foran kan flyttes tilbage så vedkommende løber samme stræk som en anden løber som er startet senere. Dette giver mulighed for grundig analyse og sammenligning af løbernes vejvalg og hastighed, både under og efter løbet.\newline
TracTrac kan stort set alt der er brug for til live visning og analyse af o-løb. Selve GPS-enheden koster 150EU, altså lige godt 1116DKK. Derudover skal der bruges et sim, data kost og enhedslicens til 88EU om året pr. enhed, hvilket vil svare til ca. 650DKK. Som det sidste skal en system-licens bruges, denne koster 990EU årligt, hvilket er 7368DKK. Dette produkt er i sådan en prisklasse, at de små amatør klubber ikke har råd til det. \citep{TTC}


\section{Opsummering}
Ud fra overstående analyse af eksisterende løsninger, mener gruppen der mangler en løsning, der er bygget på mobiltelefonens GPS. Langt de fleste mennesker har en mobil med GPS idag, hvilket vil gøre løsningen billigere, da brugeren ikke skal bruge penge på ekstra udstyr. Endomondo bruger dog mobilens GPS signal, men da der ikke en nogen replay funktion eller mulighed for at indlægge andet end Google Maps' kort bag ruten, er denne løsning ikke velegent til o-løb. 
