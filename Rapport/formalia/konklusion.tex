\chapter{Konklusion}
I gennem projektarbejdet med \textbf{\textit{IT i foreninger}}, har gruppen afgrænset til IT i o-løbsforeninger, og haft fokus på at optimere træningen for o-løbere. Gennem problemanalysen blev en række interessenter fundet, hvor der blev foretaget interview og mailkorrespondancer, for at finde frem til projektets problemstillinger. Projektets kravspecifikation blev udformet ud fra problemstillingerne, og blev afgrænset efter evner, erfaring og tid. Dette kapitel vil gennemgå om løsningsdelen overholder kravsspecifikationen og problemformuleringen.

Kravspecifikationen er opdelt i to dele. Den ene del er den grafiske afbildning, og den anden er en tekstbaseret afbildning.\newline
Programmets grafiske afbildning kan manipulere kortet, via zoom og flytning, afspille data, med pause og tempo funktioner, samt mulighed for at springe frem i afspilningen via en scrolling funktion (trackbar). Derudover findes funktionalitet til at til- og fravælge løbere, samt vælge hvilken post løberne skal samles og afspilles fra. Dette gør at programmet overholder alle krav for den grafiske afbildning. Dog er nogle kun delvist løst. Tempo-funktionaliteten er skaleret anderledes end beskrevet, ca. forhold: 1x, 4x, 9x, 16x, 25x. Brugeren har heller ikke mulighed for at tilpasse halelængden og farven på løberne. \newline
Programmets tekstbaserede afbildning viser de fleste af de ønskede informationer fra kravene. Kravet om visning af placering i løbet efter strækket og samlet tid efter strækket er ikke opfyldt. Der er blevet tilføjet information om placering for hele løbet til alle stræk.

Udover at kravspecifikationerne skal overholdes, skal projektets problemformuleringen også besvares, dette har kravspecifikationerne hjulpet med. Projektets problemformulering som er fundet frem til i problemanalysen, lød således:

Hvordan kan en telefonbaseret softwareløsning optimere evaluering og træning af o-løbere?
\begin{itemize}
	\item Hvordan kan løberne sammenlignes?
	\item Hvordan følges løberen rundt på ruten?
	\item Hvordan kan løsningen gøres brugervenlig?
\end{itemize}

OrienteeringTracker kan sammenligne løbere på to forskellige måder: Tekstbaseret, ved Data View, og grafisk ved Map View, som viser vejvalg mv. \newline
Løberen kan følges på ruten gennem Map View, og som beskrevet i kravene, kan brugeren skifte mellem poster og hastighed. Dette gør, at brugeren kan sammenligne løbernes vejvalg på ruten synkront.\newline
Efter afgrænsning fra krav til en optimal løsning, valgte gruppen at se bort fra brugervenlighed, til trods for at det er et vigtigt punkt i ethvert program.\newline
OrienteeringTracker er hidtil ikke et fuldt ud telefonbaseret softwareløsning, da programmet ikke er kørbart på en telefon. Dette skyldes, at gruppen har taget afstand fra applikationer til telefoner, og har derfor lavet et computer-program, og simuleret informationer der skulle sendes fra telefon-applikationen. Den simulerede information fra telefonen er givet ved en tredjeparts applikation.\newline
Programmet optimerer evaluering og træning af o-løbere, da det giver mulighed for sammenligning af både vejvalg, tider, hastighed og mere.




