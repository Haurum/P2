\chapter{Diskussion}
I dette kapitel vil gruppen diskutere nogle af de større beslutninger der er blevet truffet gennem projektarbejdet, specielt vil gruppen diskutere de fejl, gruppen mener, der er blevet lavet gennem projektet. 
\section{Problemanalyse}
Projektets problemanalyse er bygget meget på input fra Jens Børsting og Claus Bobach. Begge har stor erfaring inden for o-løb og indsigt i hvad den enkelte o-løbers har af behov. Gruppen mener dog, at det er svært at generalisere ud fra respondentgruppe på to personer. Gruppen havde også kontaktet andre personer, som fx. topatleter, og de fleste lovede et svar. Desværre skete dette dog aldrig, og gruppen skulle have været mere ihærdig efter at få svar fra dem, samt have skrevet til flere personer. 

Vidensdelingen med Jens var til stor gavn, men var dog svær at dokumentere, da dette tit foregik igennem uformelle Skype-samtaler og respons på det tekst gruppen har skrevet, samt programmet. Gruppen burde eventuelt have lavet et formelt interview med Jens, udover de allerede besvarede spørgsmål via en mail korrespondance. 

I begyndelsen af projektet, kontaktede gruppen Claus, for at sætte et interview op. Interviewet blev foretaget i Aalborg OK’s klubhus, hvor tre gruppemedlemmer var til stede. Gruppen havde forberedt et semistruktureret interview, som gik bedre end forventet. Det at der var tre interviewere til stede, gjorde at der kom flere uddybende spørgsmål i løbet af interviewet. Claus kom med mange gode besvarelser og kommentarer, dog var meget noget gruppen havde forventet, og gav derfor en begrænset mængde ny viden. 
\section{Program design}
Som nævnt i afsnit 6.1 om programstruktur, havde gruppen før dette projekt ikke arbejdet med større programmer i C\# eller arbejdet i Windows Form. Dette medførte, at da gruppen påbegyndte programmeringen, var det hele meget empirisk. Dette resulterede i, at der ikke var en generel struktur i programmet. Store dele af programmet blev lavet i de samme klasser, men langsomt fordelt ud på mindre klasser. I nogle af de senere OOP-kursusgange, blev gruppen præsenteret for Model-View-Controller(MVC), som er et koncept til at strukturere og opbygge et program. Programmet burde have været opbygget efter dette koncept, men eftersom at programmet var tæt på at være færdigt, da gruppen lærte om MVC, ville det kræve for meget tid at implementere.\newline
En god struktur for programmet, inden påbegyndelsen af programmeringen, vil gøre programmet lettere at programmere, og samtidig gøre det lettere for læsere at forstå kode og struktur. Det gør også at programmet bliver lettere at opdatere og rette i. 

\section{Programmeringsvalg}
I gruppens kildekode har gruppen brugt to stykker kode fundet på internettet, et til at konvertere længde- og breddegrader til UTM koordinater og et bibliotek til at læse GPX filer. Ejerne af begge stykker kode har givet lov til ubegrænset brug af deres kode. Dog ville ejeren af GPX-biblioteket gerne have sit licens dokument med i projektet. I begyndelsen af projektet følte gruppen, at det på en måde var snyd at hente kode fra nettet, som andre havde lavet. Senere i projektet følte gruppen dog det var smart, da der ikke er nogen grund til at skrive noget kode, som allerede er blevet skrevet og stillet til rådighed. \newline
Problemet ved at hente kode fra en tredjepart, er at det ikke vides hvorfor koden er skrevet som det er skrevet. Dette gør det svært at få fuld forståelse for alt koden og kan dermed give problemer hvis der opstår fejl i programmet, da det hentede kode vil være svær at debugge. Det samme kan siges om de allerede implementerede klasser og biblioteker i .NET frameworket. Gruppen mener selv, at det er i orden at bruge koden, så længe outputtet verificeres.\newline
Ved konverteringen fra længde- og breddegrader til UTM koordinater er input og output allerede kendt af gruppen. Gruppen kan dog ikke forklare den matematiske formel, eller forklare hvorfor den ser ud, som den gør. 

Gruppens program indeholder mange klasser, som gruppen selv mener kunne være skrevet sammen. Dette ville gøre programmet mere overskueligt og simpelt, eksempelvis kunne RunnerData og Runner skrives sammen, da de indeholder mange af de samme informationer. Dette kunne have været undgået hvis gruppen havde planlagt sit program, i stedet for at programmere uden at vide, hvordan det skal se ud. 

\section{Test}
Under test fandt gruppen en del uforudsete fejl, som blev rettet, hvilket har ført programmet til et tilfredsstillende stadie, med henhold til fejl i programmet. Dog ville gruppen gerne have testet GPS’en i mobilens præcision og undersøgt hvor stor præcision der kræves før programmet kunne bruges af o-løbere. Dog beskæftiger dette projekt sig kun med et bestemt kort/område. Da gruppen vurdere at GPS dataen indenfor dette område har været tilstrækkelig præcis, til at en sammenligning er mulig, har gruppen valgt ikke at udføre test af præcision i andre områder og/eller med andre mobiler og/eller med andet software. Dette vil dog være nødvendigt i videreudviklingen af programmet. Da dette er et centralt spørgsmål, fordi for stor usikkerhed på mobilens GPS vil gøre programmet ubrugelig.
\section{Opsummering}
Gruppens problemanalyse er bygget efter få kilder, det er troværdige kilder, idet både Jens Børsting og Claus Bobach har meget erfaring med o-løb. Her ville det have været mere optimalt, hvis gruppen havde forsøgt at få et svar fra de personer, der havde lovet svar, men som ikke svarede på mailkorrespondance. \newline
I forhold til gruppens design af programmet, var det meste meget empirisk, altså forsøg og forstå, og dette medførte at strukturen i programmet var meget rodet eller ikke eksisterende. Dette var desuden også skyld i at der blev afgrænset fra brugervenlighed i programmet. \newline
De valg gruppen foretog i forbindelse med udviklingen af programmet, var bl.a. at anvende kode fra tredjepart, som gruppen mente var snyd i begyndelsen, men senere indså at det ville være dumt genopfinde den dybe tallerken.
Gruppen har ikke set det nødvendigt at teste mobilens gps, i dette projekt, gruppen er dog bevidst om at det er et område der burde undersøges i videreudviklingen af programmet.


