\chapter{Perspektivering}
Dette kapitel omhandler hvilke målsætninger og planer gruppen har for projektet, hvis gruppen skulle arbejde videre med projektets eksisterende løsning.

I gruppens kravspecifikation, har gruppen i afsnit 4.1 lavet et optimalt løsningsforslag. I det afsnit er der beskrevet hvordan gruppen ønsker løsningen skulle se ud. Nedenfor er beskrevet den fremadrettede løsningsstrategi, som viser i hvilken rækkefølge gruppen ville tilføje de ønskede funktioner.

Fremadrettet skal denne løsning kunne loade kort ind dynamisk, samt danne kontrolpunkter ved at trykke på kortet, hvor posten skal være. Derefter ønsker gruppen at lave programmet om til en webservice, med et klub- og brugersystem. \newline
Gruppen vil da lave en mobil-applikation så GPX-filerne ikke skal komme fra et tredjepartsprogram. Herefter implementeres funktionalitet til live-tracking.  Mobilapplikationen skal desuden udvides, således, at den indeholder de samme funktionaliteter som webservicen. Langsigtet vil programmet blive udvidet til at kunne håndtære kamera optagelser. Dette giver en ny dimension til tilskueroplevelsen, samt giver træneren mulighed for at “se igennem løberes øjne”. \newline
Gruppen ønsker også at lave en bedre brugergrænseflade, som skal være simpel og nem at navigere rundt i. Derudover skal hjælpe-tabben implementeres, der skal hjælpe brugeren i gang med programmet, hjælpe-tabben er beskrevet i afsnit 4.1.2. 

Den optimale løsning, eller dele af den, vil kunne bruges i mange andre sammenhænge. Den vil kunne bruges i alle sportsgrene hvor man bevæger sig fra et punkt til at andet. Alle former for løb, Cykel løb, mountain bike, Adventure race, motorsport. Alle former for sejlsport og motorbådssport, roning, kajak, ridning og vædeløb af alle slags. Hertil kommer alle de aktiviteter der ikke nødvendigvis har noget med sport at gøre, men det ønskes at vide hvor folk er og hvad de ser. Politi, bederskab, falck, lastbiler, budbringere, taxa, osv. Listen er nærmest uendelig. Det behøver ikke kun at være mennesker, det kan også være maskiner eller dyr. Alt hvor der ønskes viden om position og rute.
