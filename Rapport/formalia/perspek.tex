\chapter{Perspektivering}
Dette kapitel omhandler hvilke målsætninger og planer gruppen har for projektet, hvis gruppen skulle arbejde videre med projektets eksisterende løsning.

I gruppens kravspecifikation, har gruppen i afsnit 4.1 lavet et optimalt løsningsforslag. I det afsnit er der beskrevet hvordan gruppen ønsker løsningen skulle se ud. Nedenfor er beskrevet den fremadrettede løsningsstrategi, som viser i hvilken rækkefølge gruppen ville tilføje de ønskede funktioner.

Fremadrettet skal denne løsning kunne loade kort ind dynamisk, samt danne kontrolpunkter ved at trykke på kortet, hvor posten skal være. Derefter ønsker gruppen at lave programmet om til en webservice, med et klub- og brugersystem.
Gruppen vil da lave en mobil-applikation så GPX-filerne ikke skal komme fra et tredjepartsprogram. Herefter implementeres funktionalitet til live-tracking.  Mobilapplikationen skal desuden udvides, således, at den indeholder de samme funktionaliteter som webservicen.\newline
Gruppen ønsker også at lave en bedre brugergrænseflade, som skal være simpel og nem at navigere rundt i. Derudover skal hjælpe-tabben implementeres, der skal hjælpe brugeren i gang med programmet, hjælpe-tabben er beskrevet i afsnit 4.1.2. 

Den optimale løsning vil også kunne virke for andre sportsgrene som f. eks. sejlads, skiskydning, mountainbike, cykling osv..
Tænkes der lidt længere kunne mange sportsgrene gøres mere tilskuervenlige med livetracking. Tilføjes et GoPro kamera til sportsudøveren og sender dette live sammen med livetracking, kunne der opnås en meget mere spændingsfuld oplevelse hos tilskueren.
Livetracking med live kamera-feed, kunne også bruges i militære sammenhænge. På en mission kunne missionens leder følge med i alle sine soldaters positioner under missionen og give mulighed for at se gennem den enkelte soldats øjne via kameraet. Dette ville give lederen godt overblik og dermed mulighed for at give bedre kommandoer til sine tropper.
