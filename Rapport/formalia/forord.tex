{\Huge\textbf{Forord}}
 \newline
\newline
\newline
{\Huge\textbf{Læsevejledning}}

{\Large\textbf{Kildehenvisning}}\newline
I dette projekt bruges Harvard-metoden, også kendt som Chicago-metoden, til kilde henvisning. Hvis der henvises til en bestemt kilde, efter eksempelvis en sætning, påstand eller et citat, henvises der på følgende måde: Sætning/påstand/citat [Forfatter, udgivelsesår].\newline
Hvis kilden anvendes til hele afsnit, sættes kildehenvisningen efter punktummet, således: Afsnit.[Forfatter, udgivelsesår]\newline
I afsnittet ”Litteratur” vil kilde henvisningerne blive sorteret i alfabetisk rækkefølge. Hvis det eksempelvis var en hjemmeside der blev brugt som kilde, ville det se således ud: \newline
\textbf{Kilde henvisningen fra rapporten}. Forfatter. \textit{Titel}. URL. Udgivelsesår. Dato siden er set og evt. sidetal.

\textbf{Dansk Statistik, 2008}. Dansk Statistik. \textit{Turismen – Regionalt, nationalt og internationalt}. http://www.dst.dk/pukora/epub/upload/11676/tur08.pdf, 2008. Set d. 19/11-2014 – side 8.

Hvis nogle af disse informationer mangler, eksempelvis udgivelsesår, udelades de.

{\Large\textbf{Figurhenvisning}} \newline
Igennem rapporten vil der blive henvist til figurer og illustrationer, hvor det vil blive anvist ud fra hvilket afsnit det befinder sig i, samt hvilket nummer figuren er i det omtalte afsnit. Herudover skal der være en beskrivende tekst, der forklarer figuren, eksempelvis:
\begin{flushleft}
	{\LARGE\textbf{Figur 2, afsnit 5: Figurtekst}}\newline
	Figur 5.2: Figurbeskrivelse
\end{flushleft}
