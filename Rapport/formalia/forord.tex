{\Huge\textbf{Forord}}\newline
Dette projekt er udarbejdet i samarbejde mellem fem software-studerende på andet semester fra Det Teknisk-Videnskabelige Fakultet på Aalborg Universitet. 

I udarbejdelsen af projektet, har gruppen taget udgangspunkt i Aalborg-modellen, i form af problem- og projektbaseret læring. Der tages udgangspunkt i et problem, hvor læringen sker i form af projektarbejde i grupper.

Gruppen vil gerne takke vejlederen Jacob Nørbjerg, for hans vejledning gennem hele projekt forløbet. Derudover vil gruppen gerne takke Claus Bobach for deltagelsen i interviewet, der gav indblik i o-løbstræningen, samt Jens Børsting for hans hjælp gennem hele projektet.

{\Huge\textbf{Læsevejledning}}

{\Large\textbf{Kildehenvisning}}\newline
I dette projekt bruges Harvard-metoden, også kendt som Chicago-metoden, til kilde henvisning. Hvis der henvises til en bestemt kilde, efter eksempelvis en sætning, påstand eller et citat, henvises der på følgende måde: Sætning/påstand/citat [Forfatter, udgivelsesår].\newline
Hvis kilden anvendes til hele afsnit, sættes kildehenvisningen efter punktummet, således: Afsnit.[Forfatter, udgivelsesår]\newline
I afsnittet ”Litteratur” vil kilde henvisningerne blive sorteret i alfabetisk rækkefølge. Hvis det eksempelvis var en hjemmeside der blev brugt som kilde, ville det se således ud: \newline
\textbf{Kilde henvisningen fra rapporten}. Forfatter. \textit{Titel}. URL. Udgivelsesår. Dato siden er set og evt. sidetal.

\textbf{Dansk Idrætsforbund, 2013}. Dansk Idrætsforbund. \textit{Medlemstal}.\newline http://www.dif.dk/da/om\_dif/medlemstal, 2013. Set d. 27/2-2015.

Hvis nogle af disse informationer mangler, eksempelvis udgivelsesår, udelades de.

{\Large\textbf{Figurhenvisning}} \newline
Igennem rapporten vil der blive henvist til figurer og illustrationer, hvor det vil blive anvist ud fra hvilket afsnit det befinder sig i, samt hvilket nummer figuren er i det omtalte afsnit. Herudover skal der være en beskrivende tekst, der forklarer figuren, eksempelvis:
\begin{flushleft}
	{\LARGE\textbf{Figur 2, afsnit 5: Figurtekst}}\newline
	Figur 5.2: Figurbeskrivelse
\end{flushleft}
