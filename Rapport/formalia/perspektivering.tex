\chapter{Perspektivering}
Dette kapitel omhandler hvilke m�ls�tninger og planer gruppen har for projektet, hvis gruppen skulle arbejde videre med projektets eksisterende l�sning.

I gruppens kravspecifikation, har gruppen i afsnit 4.1 lavet et optimalt l�sningsforslag. I det afsnit er der beskrevet hvordan gruppen �nsker l�sningen skulle se ud. Nedenfor er beskrevet den fremadrettede l�sningsstrategi, som viser i hvilken r�kkef�lge gruppen ville tilf�je de �nskede funktioner.

Fremadrettet skal denne l�sning kunne loade kort ind dynamisk, samt danne kontrolpunkter ved at trykke p� kortet, hvor posten skal v�re. Derefter �nsker gruppen at lave programmet om til en webservice, med et klub- og brugersystem.
Gruppen vil da lave en mobil-applikation s� GPX-filerne ikke skal komme fra et tredjepartsprogram. Herefter implementeres funktionalitet til live-tracking.  Mobilapplikationen skal desuden udvides, s�ledes, at den indeholder de samme funktionaliteter som webservicen.\newline
Gruppen �nsker ogs� at lave en bedre brugergr�nseflade, som skal v�re simpel og nem at navigere rundt i. Derudover skal hj�lpe-tabben implementeres, der skal hj�lpe brugeren i gang med programmet, hj�lpe-tabben er beskrevet i afsnit 4.1.2. 

Den optimale l�sning vil ogs� kunne virke for andre sportsgrene som f. eks. sejlads, skiskydning, mountainbike, cykling osv..
T�nkes der lidt l�ngere kunne mange sportsgrene g�res mere tilskuervenlige med livetracking. Tilf�jes et GoPro kamera til sportsud�veren og sender dette live sammen med livetracking, kunne der opn�s en meget mere sp�ndingsfuld oplevelse hos tilskueren.
Livetracking med live kamera-feed, kunne ogs� bruges i milit�re sammenh�nge. P� en mission kunne missionens leder f�lge med i alle sine soldaters positioner under missionen og give mulighed for at se gennem den enkelte soldats �jne via kameraet. Dette ville give lederen godt overblik og dermed mulighed for at give bedre kommandoer til sine tropper.
