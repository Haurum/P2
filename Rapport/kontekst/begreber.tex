\subsection{Begreber}
  
\textbf{EMIT briken} bruges til tidtagning af o-løb , som stammer fra Norge. EMIT brikken er en lille firkantet brik, på ca. 5x10x1cm, som sidder fast på løberens finger vha. en elastik. Ved hver post er en kontrol enhed, der gemmer postens nummer og tiden i brikken. Ved endt løb aflæses brikken og tiderne kan skrives ud. 

\textbf{Orienteringsløbere} eller o-løber, er en person der deltager i o-løb, uanset om det er på professionelt niveau, eller som en fritidsaktivitet.

\textbf{Stræk} er stykket mellem to poster i et orienteringsløb. Distancen for et stræk er meget individuelt, da løberne ikke nødvendigvis tager den samme rute for at nå fra en post til en anden.

\textbf{Delstræk} er mindre dele af et stræk, der ikke er fra post til post. Dette snakkes mest om hvis der er flere svæerer forhindringer på et stræk.

\textbf{Stræktider} er tiden det tog at komme fra en post til en anden.

