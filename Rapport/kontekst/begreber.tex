\newpage
\section{Begreber}
I dette afsnit vil nogle begreber og fagord inden for orienteringsløb, som bliver brugt i løbet af projektet, blive forklaret.

\textbf{Orienteringsløb}\newline
Orienteringsløb, eller forkortet o-løb, er en sportsgren der går ud på at finde vej i terræn. \newline
De typer orienteringsløb de fleste kender er det som kaldes ”lang” og ”mellem”, der indikere distancerne, hvilket vil være de discipliner der vil blive taget udgangspunkt i, i dette projekt.\newline
En lang er en distance på syv til otte kilometer, som er den normale disciplin. Mellem er derimod kortere, hvor der er flere poster og retningsskift.\newline
Kort sagt gælder orienteringsløb om at finde en række poster i et terræn, som kunne være en skov, vha. et kort og et kompas. Et af de vigtige elementer i orienteringsløb er kortet. Kortet fremstilles af de lokale orienteringsklubber og deres medlemmer, vha. eksisterende kort, luftfotos og laserscannede højdekurver. Løberne skal kunne aflæse kortet, for at finde det hurtigste og sikreste vejvalg mellem punkterne, da det ud fra kortet er muligt, at læse hvordan terrænet ser ud. \citep{DOF}   

\textbf{EMIT brik systemet}\newline
Til tidtagning af o-løb bruges oftest EMIT brikken, som stammer fra Norge. MITEMIT brikken er en lille firkantet brik, på ca. 5x10x1cm, som løberen har i sin håndflade. Ved hver post er en kontrol enhed, der gemmer postens nummer og tiden i brikken. Ved endt løb aflæses brikken og tiderne kan skrives ud. 

\textbf{Orienteringsløbere}\newline
En orienteringsløber, eller o-løber, er en person der deltager i o-løb, uanset om det er på professionelt niveau, eller som en fritidsaktivitet.

\textbf{Stræk}\newline
Et stræk er stykket mellem to poster i et orienteringsløb. Dette er oftest meget individuelt, da løberne ikke nødvendigvis tager den samme rute for at nå fra en post til en anden.

\textbf{Delstræk}\newline
Mindre dele af et stræk, der ikke er fra post til post. 

\textbf{Stræktider}\newline
Tiden det tog at komme fra en post til en anden.\newline

