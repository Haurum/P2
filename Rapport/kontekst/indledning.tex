\chapter{Indledning}
Der findes mange forskellige typer af foreninger i Danmark, hvor nogle er større end andre, og derfor har flere ressourcer til rådighed. De mindre foreninger skal derfor være mere ressource bevidste end andre, og arbejdsbyrden på de enkelte medlemmer kan være stor.\newline
Denne rapport vil forholde sig til orienteringsløb i mindre foreninger, hvor der i Danmark er 76 foreninger med omtrent 7.000 medlemmer \citep{DIF}. Igennem et interview med Jens Børsting, som har løbet og været engageret i o-løb i over 30 år, blev det konkluderet, at der er meget arbejde for medlemmerne i o-løbsforeninger, og at det er begrænset hvor meget software der findes til at hjælpe o-løbere. \newline
I et orienteringsløb skal en løber med kort og kompas hurtigst muligt finde et antal forudbestemte poster, typisk i en skov. Da løberne undervejs er svære at holde under opsyn, er det ikke attraktivt at være tilskuer til et o-løb. Der er samtidigt ikke mange unge o-løbere til trods for, at Danmark har nogle af verdens bedste o-løbere \citep{RANK}. \newline
Hidtil har posterne været opsat fysisk, og ruten der er tiltænkt løbet er planlagt på forhånd. Dette arbejde indebærer flere timers arbejde både før og efter en træningsgang eller løb, da posterne også skal hentes efterfølgende. Derudover kan det være svært at sammenligne de vejvalg, den enkelte løber har taget på ruten, da sammenligning kun kan ske ud fra løbernes hukommelse. Dette gør det svært for træneren at fortælle hvad løberen kunne have gjort anderledes, medmindre træneren løber efter løberen, hvilket vil sige at løberen har spotter på. Løberen kan heller ikke sammenligne sine vejvalg med de andre løbere, for at finde svagheder i sit eget løb.\newline 
For amatør o-løbere er det svært at udvikle sig, grundet den begrænsede feedback løberen kan få fra træningen.\newline
Den initierende problemstilling i dette projekt lyder derfor sådan:\newline
Hvordan kan planlægningen, afviklingen og opfølgningen af træningen for amatør o-løbere forbedres/effektiviseres vha. en IT-løsning?
\begin{itemize}
	\item Hvordan foregår en o-løbstræning?
	\item Hvilke problemer forekommer der under en o-løbstræning?
	\item Hvilke teknologiske redskaber bruges til en o-løbstræning?	
\end{itemize} 


