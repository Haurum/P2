\chapter{Problembeskrivelse}
I dette kapitel vil problematikkerne i denne problemanalyse blive beskrevet, og herefter vil rapporten afgrænses til et delområde, hvoraf en problemformulering vil uddrages.

\section{Problemafgrænsning}
I udarbejdelsen af problemanalysen blev der fundet to problematikker ved o-løbstræninger. Den første omhandler opsætningen af poster, normale og elektroniske, da trænerne bruger flere timer på at forberede en træning. Posterne skal både ud før træning og samles ind igen efter, hvor de elektroniske poster er mere tidskrævende end de små poster, kun bestående af en lille skærm. Udover dette, er arbejdskraften typisk frivillig, så trænere og andre klubmedlemmer må satse på, at en person frivilligt vælger at bruge tid på forberedelsen. En anden problemstilling er evaluering af løbere under træning. Problematikken opstår i det, at evalueringen kun kan ske ved, at to løbere sætter sig ned og kigger på kortet over den udførte rute, og prøver at erindre hvilke vejvalg de har taget. De har ikke nødvendigvis taget den samme rute, trods de har en næsten ens samlet tid på løbet, hvor den hurtigste rute er muligvis en kombination af de to ruter. Gruppen har valgt at afgrænse sig til evaluering af træningen for o-løbere, da gruppen ser størst potentiale i dette. Dog ser gruppen også en mulighed for at integrere en løsning til opsætning af poster, ved brugen af mobilens GPS. 

Der findes flere eksisterende løsninger, hvor mange anvender GPS-ure, men som beskrevet, er ret dyre. Problematikken er, at der ikke kan sammenlignes vejvalg og tider på et o-løbskort, da dette skal gøres efterfølgende på computeren. Da Endomondo anvender Google Maps, vil o-løberen blot være i en grøn skov på et kort, så løberen kan ikke se andet end en rute på en grøn plet. EMIT brikker anvendes til tidtagning under træningen, men denne løsning er ikke særligt anvendelig, da den blot giver stræktider, så løbere kan kun evaluere hvor hurtig de er mellem to stræk, men ikke vejvalg. QuickRoute er et godt værktøj til evaluering af løberen, men da denne løsning kræver et GPS-ur, bliver det hurtigt dyrt for en amatør o-løber.
TracTrac er en allerede kendt løsning, men som beskrevet i afsnittet omhandlende løsningen, skal brugere have licens, samt firmaets egne GPS-enheder, hvilket igen bliver for dyrt for de mindre foreninger.

Ud fra problemanalysens resultater, har gruppen konkluderet, at en software løsning skal bruge mobilens GPS, og derved være en billigere løsning end eksempelvis TracTrac. Løsningen skal også kunne indsætte o-løbs kort, så løberne kan se præcist hvor de har været i terrænet, i stedet for at bruge Google Maps som Endomondo. 

Dette kan optimeres betydeligt med en telefonbaseret softwareløsning, hvor der i afsnittet om eksisterende løsninger er blevet beskrevet, at denne løsning skal kunne hjælpe til evaluering af træningen, uden brug af et GPS-ur.

\section{Problemformulering}
Ud fra ovenstående afgrænsning er følgende problemformulering blevet udarbejdet:

Hvordan kan en telefonbaseret softwareløsning optimere evaluering og træning af o-løbere?
\begin{itemize}
	\item Hvordan kan løberne sammenlignes?
	\item Hvordan følges løberen rundt på ruten?
	\item Hvordan kan løsningen gøres brugervenlig?
\end{itemize}
