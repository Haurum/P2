\section{Løsningsstrategi}
I dette afsnit vil planen for programmet blive beskrevet, med en overordnet beskrivelse af løsningsstrategi, samt en beskrivelse af valg mht. værktøjer brugt i udviklingen af programmet.

I henhold til gruppens kravspecifikation, vil GPS dataen til programmet blive simuleret af en app til en smartphone. Gruppen har brugt en app kaldet, myTracks til dette formål. myTracks optager og sender GPS signaler i form af en GPX-fil, som bliver tilsendt via e-mail til brugeren.
Som beskrevet i kravspecifikationen, skal brugeren sørge for at kortet anvendt skal være geostationært. Dette betyder, at der sammen med kortet medfølger en tekstfil med tal, der beskriver kortets placering på jordkloden. Med denne fil kan koordinater beskrevet med UTM-systemet (beskrives i teoriafsnittet), oversættes til pixels, og derved gøre det muligt at tegne afstandstro GPS-ruter på kortet. I dette projekt anvendte gruppen QLandkarteGT til at geostationære et kort, som blev anvendt til udvikling af programmet.

myTracks anvender koordinater i Lat/Long-formatet, hvor det geostationære kort oversætter UTM-koordinater til pixels. Dette betyder, at en konvertering fra Lat/Long til UTM er nødvendig, denne konvertering vil blive beskrevet i teoriafsnittet. 

Programmet vil her bestå af to dele, hvor den første del vil behandle data i form af pixels til at tegne ruter på kortet. Anden del af programmet vil bruge UTM-koordinater til at udregne data over løberne, som f.eks. hastighed og distance.

Til udviklingen af dette program har gruppen valgt at benytte Visual Studios Windows Forms, da gruppen ønsker at have en grafisk brugergrænseflade.