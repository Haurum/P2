\chapter{Teori} 
I dette kapitel vil de anvendte teorier blive beskrevet. Herunder længde- og breddegrader, Universal Transverse Mercator systemet og konverteringen mellem dem. 

\section{Længde- og breddegrader}
I et geografisk koordinatsystem, er jorden inddelt i længde- og breddekredse. Idet jorden er rund, er disse længde- og breddekredse målt i grader, og de bliver i stedet kaldt længde- og breddegrader. \newline
Længdegrader beskriver de vertikale linjer på verdenskortet på figur 5.1, som går fra nordpolen til sydpolen. Disse bliver også betegnet som meridianer. Meridianen der går gennem Greenwich kaldes nulmeridianen, hvorefter den første vestlige længdegrad bliver udtrykt som 1\textdegree vestlig længde, og det samme gælder for den første østlige længdegrad, som vil være 1\textdegree østlig længde. Disse længdegrader bliver herefter talt op, i hver deres retning. Idet en cirkel er 360\textdegree, er de sidste meridianer på hver sin side af verdenskortet kaldet 180\textdegree østlig/vestlig længde. Det skal desuden også nævnes, at længdekredsene ikke nødvendigvis har den samme afstand, da længdekredsene har mindre indbyrdes afstand jo tættere på polerne de er. \newline
Breddegrader beskriver de horisontale linjer i verdenskortet på figur 5.1. Disse linjer er parallelle med ækvator, og deres indbyrdes afstand bliver derfor hverken mindre eller større, men deres omkreds bliver kortere alt efter hvor tæt på polerne de befinder sig. \citep{LongLat}

\begin{figure} [h]
	\centering
	\includegraphics[width=.5\textwidth]{billeder/longlatmap}
	\caption{Verdenskort med længde- og breddegrader}
\end{figure}

\section{Universal Transverse Mercator systemet}
Universal Transverse Mercator systemet, forkortet UTM, er en anden måde at finde lokationer på jordkloden. Denne metode gør brug af, at jordkloden, som reelt set er rund, bliver omformet til en flade, ligesom et stykke papir. Her bliver jorden inddelt i 60 zoner, hvor hver zone er opdelt fra øst til vest, med numre, og syd til nord med bogstaver. De østlige og vestlige, startende fra 180 meridianen, regnes fra vest til øst med 6\textdegree mellemrum. Enkelte zoner er dog blevet gjort bredere eller smallere. Hver zone er nummereret efter deres placering fra vest.\newline
Nordligt og sydligt er inddelt mellem 80\textdegree S bredde, og 84\textdegree N bredde, med en afstand på 8\textdegree, udover en enkelt zone der er 12\textdegree. Hver zone er  navngivet ved et bogstav, fra C-X, startende med C i den sydligste zone.
Alle zonerne i UTM systemet er baseret på en Transverse Mercator projektion, som gør det muligt at kortlægge en region uden stor forvrængning af jordklodens ellers runde form. Hvis jorden blev kortlagt i én stor region, vil dette forårsage store afstands fejlberegninger, da der ikke vil blive taget højde for jordklodens runde form. Det er her at Transverse Mercator projektionen er smart, da den sørger for en lav grad af forvrængning, da det er smalle breddegrader (6\textdegree) der typisk anvendes til zonerne. \newline
Ved positionering gennem UTM systemet, oplyses følgende: Først en zone fra UTM, som f. eks. 17T. Dernæst en østlig og en nordlig afstand. Den østlige udregnes ud fra den centrale meridian. Den centrale meridian er den længdegrad som går igennem midten af zonen. Denne er sat som 500.000 m østlig for at undgå negative tal, dvs. et punkt vest for meridianen giver et tal under 500.000 m, hvor et tal øst for vil være over 500.000 m.  Den nordlige afstand beregnes på følgende måde: Hvis punktet er på den nordlige halvkugle, beregnes den nordlige afstand som afstanden til ækvator. Hvis punktet er på den sydlige halvkugle, beregnes den nordlige afstand som 10,000,000 m minus afstanden til ækvator, for at undgå negative tal. \citep{UTMTeori}

\section{Konvertering}
Som beskrevet i afsnittet løsningsstrategi, er en konvertering mellem længde- og breddegrads systemet til UTM systemet nødvendig. Dog er udregningen ret kompleks og gruppen har derfor valgt ikke at redegøre for formlen i dette teoriafsnit.

Udover konverteringen fra længde- og breddegrads systemet til UTM, skal en konvertering fra UTM til pixels også bruges i projektet. Dette sker ved brugen af en world file. World filen bruges til at oversætte UTM koordinater til pixels på kortet, så det GPS data der loades ind, kan findes på kortet. Dette projekts world file ser således ud:
\begin{description}
	\item[mpx] 0.84964441
	\item[roty] 0.00000000
	\item[rotx] 0.00000000
	\item[mpy] -0.84964441
	\item[x0] 539276.35483168
	\item[y0] 6250863.73506770
\end{description}
\textbf{mpx} - Antal meter pr pixel i x-retning\newline
\textbf{roty} - Rotationsværdi om y-aksen\newline
\textbf{rotx} - Rotationsværdi om x-aksen\newline
\textbf{mpy} - Antal meter pr pixel i y-retning (oftest negativ)\newline
\textbf{x0} - X koordinatet for midten af pixlet i øvereste venstre højrne\newline
\textbf{y0} - Y koordinatet for midten af pixlet i øvereste venstre højrne\newline
Da o-løbskort aldrig er roteret, vil roty og rotx altid være 0, og der vil kigges bort fra disse værdier.

Grunden til at mpy oftest er negativ, er fordi pixels begynder fra øverste venstre hjørne, hvor et UTM koordinatsystemet begynder fra nederste venstre hjørne.Tages der udgangspunkt i figur 5.2 kan det forklares således:

\begin{figure} [h]
	\centering
	\includegraphics[width=.5\textwidth]{billeder/UTMtilPIXEL}
	\caption{Figur til forklaring af UTM til pixel}
\end{figure}

Koordinatsættet (0,0) vil svare til (x0,y0).
Koordinatsættet (1,0) vil svare til (mpx + x0,y0)
Koordinatsættet (0,1) vil svare til (x0,mpy + y0)
Koordinatsættet (1,1) vil svare til (mpx + x0, mpy + y0)

For at finde pixel-position (x’, y’) ud fra UTM koordinat (x,y) bruges følgende formel:\newline
x’ = (x - x0) / mpx\newline
y’ = (y - y0) / mpy\newline

\section{Opsummering}
Længde- og breddegrader og Universal Transverse Mercator systemet er blevet kort forklaret. Gruppen har valgt ikke at forklare den brugte formel i konverteringen fra længde- og breddegrader til UTM koordinater. Det lykkedes at opstille formlen for konverteringen fra UTM koordinater til pixels. Denne teori vil blive brugt i kapitlet “Implementeringen”. 