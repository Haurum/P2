\chapter{Test}
Gruppen har foretaget nogle tests i forbindelse med udviklingen af programmet, disse metoder vil blive beskrevet herunder, samt hvilke software og metoder gruppen har anvendt for at teste programmet. De anvendte testtyper begrundes ved at brugeren ikke skal kunne give forkert input eller uønsket kunne manipulere med data.

\section{Monkey testning}
Denne test-type kan kategoriseres på forskellige måder, da der er forskellige typer af monkey testning. Der er Smart monkey test, og Ignorant monkey test (også kaldet Dumb monkey test), hvor begge typer er automatiske tests udført af software. De kategoriseres således:
Smart monkey test: Her gøres der brug af unit test, da der laves test-suites, hvor ”aben” kender til input-type og ved hvilke metoder der skal testes. Denne test-type skal oftest købes, da det er mere avanceret software, som integreres med den kode som skal testes.\newline
Smart monkey tests er hurtig til at finde fejl i programmet, hvis det er sat ordentligt op. Fejl fundet ved denne test-type er fra store til små, og kan typisk findes i en log-fil efter testen er kørt.\newline
Ignorant monkey test: Denne test-type er en blackbox test, da den ikke har viden om programmets kode, og derfor tester random inputs på enhver mulig måde. Ignorant monkey tests kan findes som gratis software, bl.a. ”GUI tester”, som anvendes i dette projekt. ”Aben” kræver tæt på ingen opsætning af program for at kunne køre, og laver stress-test af programmet, og kan køre i flere dage, hvis ikke den finder fejl der får programmet til at crashe. Ved fejl i denne test, skal testeren typisk sidde og overvåge testen, da der typisk ikke er beskrivende log-filer over testen efter kørsel. Ignorant monkey test finder typisk store fejl i programmet, hvor de mindre fejl i koden ikke opdages. \citep{ExI}\newline
Gruppen har valgt at anvende denne test-type, for at sikre, at brugeren ikke har mulighed for at intaste noget forkert input.

\subsection{GUI tester}
GUI tester er et simpelt ignorant monkey test software, udviklet af Luigi Poderico, som er anvendt i dette projekt. Programmet køres ved at have det testede program i forgrunden, og derefter klikke på højre CTRL knap. GUI testeren vil herefter klikke på forskellige steder i programmet, og indtaste forskellige inputs hvis muligt. Problematikker ved dette software er, at det kan åbne op for andre programmer, samtidig med at at test-softwaret køres, og derved prøve at ”teste” de andre programmer. \citep{GUItester} \newline
Ved brug af GUI tester i dette projekt, blev OrienteeringTracker startet, og GUI testen sat i gang. Der blev lavet følgende tests:

1: Test uden at have manipuleret programmet. Der er her ikke loadet noget ind i programmet, og programmet består derfor kun af et kort, Map View tab, Data View tab og Load-knap. \newline
2: Test ved tryk af Load-knap, så dialogboksen åbnes, og testen derfor kan vælge hvilken som helst mappe.\newline
3: Test efter programmet har loadet filer. Her er både løbere, kort, poster og data tilstede i programmet.\newline
I første testcase åbnes programmet, og GUI test startes. ”Aben” trykker mange steder, og ender med at minimere programmet flere gange før den får trykket på ”Load”-knappen. Der er ikke sket nogle fejl i programmet på dette stadie.\newline
I anden testcase har ”aben” klikket på Load, får valgt en mappe som ikke indeholder de rigtige filer, og programmet kaster en ArgumentNullException. Løsningen på denne fejl er at tjekke om mappen med filerne indeholder filtyper der ender på .gpx og utm. \newline
I tredje testcase er de rigtige filer loadet ind i programmet, og ”aben” får lov til at teste både Map View og Data View. Den skifter selv mellem tabs, og prøver, i Map View, at indsætte forskellige værdier i både Tempo og Startpoint. Disse to værdier bliver derved sat til henholdsvis det mindst mulige (ved for lav værdi) eller det højest mulige (hvis værdien indsat er højere end muligt for programmet). Den slår nogle af løberne fra i checkboksen, og får både prøvet at starte løbet, og pauser løbet. I Data View valgte den en celle, og skrev en værdi i cellen. Dette skulle ikke ske, da en bruger så ville kunne manipulere med det viste. Dette blev ændret ved at gøre cellerne ReadOnly. ”Aben” får både klikket sig ind på et stræk, og gjort brug af tilbage-knappen.

\section{Blackbox testning}
Blackbox testning er en test-type, hvor testeren indtaster input til programmet, og ser på hvorvidt outputtet er korrekt. I dette projekt vil denne test-type bruges til at se, hvorvidt filer der loades ind håndteres korrekt. Her vil testcases indkludere følgende:

Testcase 1: Indlæsning af flere af samme person, for at observere hvorvidt Data View tabbet giver korrekt information, og hvordan flere personers rute afbildes i Map View. Her forventes, at Data View kan sorteres per kolonne, hvor en passende sortering finder sted, og at flere versioner af samme person kan afbildes i Map View uden exceptions.\newline
Testcase 2: Indlæsning af personer, som ikke når alle poster på ruten. Her forventes at deres data bliver sat til 0 eller ”X”.\newline
Testcase 3: Lukke dialogboksen uden valg af mappe, efter tryk på Load-knappen.\newline

I testcase 1, hvor flere af samme person læses ind, bliver Map View vist korrekt, og alle ”personer” af samme persons rute, blev lagt ovenpå hinanden. I Data View blev data vist korrekt, men sortering af position fejlede. De øverste positioner ved sortering blev: 1, 10, 11, 12... 2, 21, 22... Dette skyldes, at sorteringen skete ved sammenligning af strenge, og ikke ved tal. Dette er blevet rettet, således at det kommer i den rigtige rækkefølge.\newline
I testcase 2 blev personen, som ikke nåede alle poster, håndteret som forventet. Tider blev udskiftet med ”X” i de tilfælde personen ikke nåede en post, og difference i tid, samt distance løbet, blev sat til 0. Løberen indgår med korrekte tider i de stræk, som løberen nåede.\newline
I testcase 3 crashede programmet med en ArgumentOutOfRangeException. Denne fejl blev rettet, ved at der ikke bliver loadet noget ind i programmet, før der bliver valgt en mappe og trykket “OK” i dialogboksen.

\section{Unit testning}
Unit testning er en metode at teste software på, ved at teste forskellige dele af programmet individuelt. I objektorienteret programmering, som softwaren til dette projekt er lavet efter, vil opdelingen typisk ske på klasseniveau. Fordelene ved at teste de enkelte klasser individuelt, er den sikkerhed det giver, både da funktionaliteten af delene garanteres at virke, men også at den videre udvikling af programmet bliver lettere. Skal koden på et tidspunkt skrives om på grund af problemer med fx ydeevne, er det let at garantere, at programmet ikke pludselig holder op med at virke, så længe den del af programmet stadig kan gennemføre sine tests.
Tests bliver på den måde også en form for dokumentation over hvilken funktionalitet en klasse eller funktion i programmet har. Hvis en anden programmør end den oprindelige skaber af koden, skal bruge en funktion, kan vedkommende kigge på de test der er skrevet, for at se hvad det forventede output er, og hvordan den del opføre sig i forhold til de forskellige inputs der måtte være. På den måde kan en anden programmør hurtigt finde ud af hvordan den del af koden bruges, fremfor at skulle bruge lang tid på at tyde koden, og forestille sig alle tænkelige scenarier der måtte værre. 

Af Unit-tests er blevet lavet 3 tests. 
Den ene tester CalcSingleLength på Helper klassen. CalcSingleLength, beregne grundlæggende bare distancen mellem to punkter, så derfor er testen udført ved at tjekke om funktionen returnere 5, som er distancen mellem punkterne (-2,1) og (1,5).

\begin{lstlisting}
[TestCase]
public void CalcSingleLength_TwoPoints_Return5()
{
    double dist = Helper.CalcSingleLength(-2, 1, 1, 5);
    Assert.AreEqual(dist, 5);
}
\end{lstlisting}

Derefter er ConvertLatLongToUTM funktionen testet, ved at sammenligne med et andet konverteringsprogram, på nettet. \citep{LatLongConvert}  

\begin{lstlisting}
[TestCase]
public void ConvertLatLongtoUTM_LatLong_UTM()
{
    double easting;
    double northing;
    string zone;
    Helper.ConvertLatLongtoUTM(57.121332, 9.953613, out northing, out easting, out zone);

    Assert.AreEqual(Math.Round(easting, 2), Math.Round(557740.21, 2));
    Assert.AreEqual(Math.Round(northing, 2), Math.Round(6331295.72, 2));
    Assert.AreEqual(zone, "32V");
}
\end{lstlisting}

ReadControlPoint er blevet testet, ved at kalde funktionen med linjen "539446.2;6249967;200;1056". Der testes derefter, om pixelkoordinatet på den pågældende instans af ControlPoint, er blevet sat til (200,1056).

\begin{lstlisting}
[TestCase]
public void ReadControlPoints_LineAndNr_PopulatedCP()
{
    ControlPoint cp = new ControlPoint();
    string line = "539446.2;6249967;200;1056";
    int nr = 1;
    cp.ReadControlPoint(line, nr);

            
    PointF point = new PointF(200,1056);
    Assert.AreEqual(cp.Cord.pixelPoint, point);
}
\end{lstlisting}


