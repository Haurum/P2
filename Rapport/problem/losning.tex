\chapter{Kravspecifikationer}

\section{Optimale løsningsforslag}
Den ideelle løsning vil bestå af to dele, en mobiltelefon med GPS og en website/server som kan vise dataene sendt fra mobilen

Mobilen skal kunne:
\begin{itemize}
	\item Måle, sende og optage GPS positionen på løberen undervejs på ruten.
	\item Tilslutte sig en bestemt bane, som træneren har opsat og lagt på serveren inden træning.
	\item Mobilen må ikke hjælpe løberen undervejs i løbet, med at vise positionen på ruten.
\end{itemize}

Websitet/serveren skal kunne:
\begin{itemize}
	\item Opsætte baner inden løbet.
	\item Vise løbernes position på kortet.
	\item Vise diverse statistik og data for løberens tur på banen. Dette vil være tider, afstande og hastigheder, samt en grafisk visning af den rute løberen har løbet.
	\item Sammenligne to løberes tur på samme rute, eller sammenligne med banens gennemsnit i forhold til statistik/rådata.
	\item Vise et grafisk replay af den rute løberen har løbet, med mulighed for at afspille flere løbere samtidig og dermed sammenligne vejvalg.
	\item Være kompatibelt med GPS-ure.
\end{itemize} 

Mobil-appen vil have en simpel brugergrænseflade hvor brugeren vil have mulighed for at vælge den bane vedkommende skal løbe. Herefter vil brugeren have mulighed for at trykke start, hvorefter mobilen kun vil vise hvor lang tid der er brugt indtil videre og en afslut knap. Under løbet sender mobilen løbende data om GPS position og tiden siden start. Når turen er løbet færdig trykker brugere på afslut og appen vil sende de sidste data og vise nogle resultater. Dette kan eksempelvis være tid i alt, højeste hastighed, gennemsnitshastighed, men også data om løberen sammenlignet med andre på samme bane. Dette kan f. eks. være hvor mange sekunder efter den hurtigste vedkommende har brugt på de enkelte stræk.

Inden løbet skal træneren eller arrangøren, kunne uploade den bane som personen har lavet, til serveren. Banerne kan vælges fra appen på mobilen, af de forskellige løbere. Efter løbet skal brugerne som sagt kunne uploade sine resultater til websitet/serveren fra mobil appen, under og efter et endt løb. Disse resultater skal så kunne sammenlignes med de andre løbere, der har løbet den samme bane. Ud over at sammenligne statistik over løberens rute, som beskrevet ovenfor, skal den også kunne vise grafisk den rute som løberen har løbet. Derudover skal der være en replay funktion, så de vejvalg som løberen tager, kan ses. Så kan der indsættes flere løbere på den grafiske visning, så flere løberes vejvalg kan sammenlignes.

\section{Gruppens løsningsforslag}
\subsection {Løsningsstrategi}
Løsningsstrategi til statusseminariet.
\begin{itemize}
\item Først vil gruppen undersøge om det er muligt, med gruppens kompetencer i C\#, at kunne dynamisk importere, vise og bruge o-løbskort til løsningen.
\begin{itemize}
\item Hvis overstående ikke er muligt, vil gruppen forsøge at simulere importen af o-løbskortet, og derved kun arbejde ud fra et kort.
\end{itemize}
\item Hvis et af overstående lykkes, vil gruppen efterfølgende lave forskellige former for grafisk analyse, i form af en replay funktion på kortet. Først med kun en løber, og derefter to til flere løbere, for at kunne sammenligne løberne med hinanden.
\item Hvis de overstående fejler, vil gruppen fokusere på detaljerede tekstbaseret analyse, i form af tider, afstande, hastighed osv. som alle skal kunne sammenlignes mellem andre løbere.
\item Hvis alt andet virker for svært eller uinteressant, har gruppen tænkt på at udarbejde en analyse af nøjagtigheden i mobiltelefonens GPS.
\end{itemize}
Gruppen har valgt at simulere mobil GPS dataene, frem for at bruge tid på at lave en mobilapplikation. I tilfælde af at gruppen skal lave analyse af nøjagtigheden i mobiltelefonens GPS, vil mobil GPS dataene ikke blive simuleret. 
 
Hvis der er tid til overs, vil gruppen bruge muligheden til at implementere virtuelle poster eller delvist virtuelle poster. Med virtuelle poster menes at, at vi med GPS enheden konkludere om løberen har besøgt en bestemt post, og på en eller anden måde giver signal til løberen om dette. Med delvist virtuelle poster menes at, der vil blive sat en fysisk skærm ud på banen for at markere posten som sædvanligt, resten vil fungere som den virtuelle post.




